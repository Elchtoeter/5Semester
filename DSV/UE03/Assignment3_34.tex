\documentclass[12pt,a4paper,austrian]{article}
\usepackage{graphicx}
\usepackage[austrian, english]{babel}
\usepackage[utf8]{inputenc}
\usepackage{listings}
\usepackage{multirow}
\usepackage{epstopdf}
\usepackage{amsmath}
\usepackage{amssymb} % fuer Mengen \N, Q, C, R
\graphicspath{{./fig/}}


%% Satzspiegel
\setlength{\hoffset}{-1in} \setlength{\textwidth}{18cm}
\setlength{\oddsidemargin}{1.5cm}
\setlength{\evensidemargin}{1.5cm}
\setlength{\marginparsep}{0.7em}
\setlength{\marginparwidth}{0.5cm}

\setlength{\voffset}{-1.9in}
\setlength{\headheight}{12pt}
\setlength{\topmargin}{2.6cm}
   \addtolength{\topmargin}{-\headheight}
\setlength{\headsep}{3.5cm}
   \addtolength{\headsep}{-\topmargin}
   \addtolength{\headsep}{-\headheight}
\setlength{\textheight}{27cm}

%% How should floats be treated?
\setlength{\floatsep}{12 pt plus 0 pt minus 8 pt}
\setlength{\textfloatsep}{12 pt plus 0pt minus 8 pt}
\setlength{\intextsep}{12 pt plus 0pt minus 8 pt}

\tolerance2000
\emergencystretch20pt

%% Text appearence
% English text
\newcommand{\eg}[1]%
  {\selectlanguage{english}\textit{#1}\selectlanguage{austrian}}

\newcommand{\filename}[1]
  {\begin{small}\texttt{#1}\end{small}}

\newcommand\IFT{\unitlength1mm\begin{picture}(10,2) \put (1,1)
{\circle{1.7}} \put(2,1){\line(1,0){5}} \put(8,1)
{\circle*{1.7}}\end{picture}}
\newcommand\FT{\unitlength1mm\begin{picture}(10,2) \put (1,1)
{\circle*{1.7}} \put(2,1){\line(1,0){5}} \put(8,1)
{\circle{1.7}}\end{picture}}

% A box for multiple choice problems
\newcommand{\choicebox}{\fbox{\rule{0pt}{0.5ex}\rule{0.5ex}{0pt}}}

\newenvironment{wahrfalsch}%
  {\bigskip\par\noindent\makebox[1cm][c]{richtig}\hspace{3mm}\makebox[1cm][c]{falsch}
   \begin{list}%
   {\makebox[1cm][c]{\choicebox}\hspace{3mm}\makebox[1cm][c]{\choicebox}}%
   {\setlength{\labelwidth}{2.31 cm}\setlength{\labelsep}{3mm}
    \setlength{\leftmargin}{2.61 cm}\setlength{\listparindent}{0pt}
    \setlength{\itemindent}{0pt}}%
  }
  {\end{list}}

\newcounter{theaufgabe}\setcounter{theaufgabe}{1}
\newenvironment{aufgabe}[1]%
  {\bigskip\par\noindent\begin{nopagebreak}
   \textsf{\textbf{\arabic{theaufgabe}.\thinspace Aufgabe}}\quad
      \textsf{\textit{#1}}\\*[1ex]%
\stepcounter{theaufgabe}\hspace{2ex}\end{nopagebreak}}
  {\par\pagebreak[2]}

% Innerhalb der Aufgaben erfolgt die weitere Unterteilung mittels einer
% enumerate Umgebung, die allerdings a), b),... zaehlen soll.
\renewcommand{\labelenumi}{\alph{enumi})}
\renewcommand{\labelenumii}{\arabic{enumii})}

% A box to tick for everything which has to done
\newcommand{\abgabe}{\marginpar{$\Box$}}
% Margin paragraphs on the left side
\reversemarginpar

% Language for listings
\lstset{language=Vhdl,
  basicstyle=\small\tt,
  keywordstyle=\tt\bf,
  commentstyle=\sl}

% No indention
\setlength{\parindent}{0.0cm}
% Don't number sections
\setcounter{secnumdepth}{0}


%% Beginning of the text

\begin{document}
\selectlanguage{austrian}
\pagestyle{plain}
% This is the header section
  \thispagestyle{empty}
  \noindent
  %\includegraphics[height=2.5cm]{fig/JKULogoFullEnglShort}
  % blagOPP
  \begin{minipage}[b][2.4cm]{1.0\textwidth}  
  \begin{tabular}{l p{11cm} r} 
    \multicolumn{3}{c}{\centering \begin{large}\begin{bf}
  	\textsf{Digitale Signalverarbeitung, WS 2019/20} \end{bf}\end{large} }  
  	 \\
  	\multirow{2}{*}{\includegraphics[height=1.6cm]{fig/JKU_Logo}} 
  	& \centering Bernhard Fürst k0442418, Sebastian Ortner k01607533, Gruppe 34 \vspace{1.3em}  &
    \multirow{2}{*}{\includegraphics[height=1.9cm]{fig/ISP-Logo-color-02}}  \\	
    & \centering \textit{3. Übung} & \\     
    \multicolumn{3}{c}{\centering \begin{large}
    \textit{Abtastung, DTFT und Frequenzanalyse}%
    \end{large} }  
 
  \end{tabular} 
  \end{minipage}
%  \vspace{-1.2em}

  \noindent \rule[0.8em]{\textwidth}{0.12mm}\\[-0.5em]

%%%%%%%%%%%%%%%%%%%%%%%%%%%%%%%%%%%%%%%%%%%%%%%%%%%%%%%%%%%%%%%%%%%%%%%%%%%%%%




\begin{aufgabe}{}
  \begin{enumerate}
    \item Welches Verhalten beobachten Sie?

          Der Minutenzeiger scheint rückläufig zu sein.
    \item Mit welcher Frequenz wird die Uhrzeit abgetastet?
    
          Durch das \textit{pause} Kommando wird die Zeit zwischen jeder Iteration des Skripts auf 100 ms festgelegt. 
          Betrachten wir die Abtastung als das Neuzeichnen des Plots so entsprechen diese 100 ms der Abtastfrequenz. 
          Die Frequenz des "`Minutenzeigers" beträgt in diesem Fall $100*\frac{12}{11}$ ms.  
          Das Verhältnis $\frac{f_s}{f_{min}}$ entspricht in jedem Fall immer $\frac{11}{12}$.

    \item Was ist die Ursache für dieses Verhalten?
    
    Die Ursache für dieses Verhalten ist ein zeitlicher Aliasingeffekt auch Stroboskopeffekt genannt. Liegt die Signalfrequenz zwischen der Nyquist-Frequenz und der Abtastfrequenz, so scheint der Zeiger rückwärts zu laufen
  \end{enumerate}
\end{aufgabe}

\begin{aufgabe}{}
  \begin{enumerate}
    \item Was fällt Ihnen in Bezug auf die Periodizität auf ?
    
    Bei allen Vielfachen $n * 2\pi | n\in \mathbb{Z}$ wiederholt sich das Signal.

    \item Was fällt Ihnen beim Phasenverlauf auf?
    
    Bleibt konstant bei 0 

  \end{enumerate}
  
\end{aufgabe}


\end{document}
