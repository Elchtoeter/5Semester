\documentclass[12pt,a4paper,austrian]{article}
\usepackage{graphicx}
\usepackage[austrian, english]{babel}
\usepackage[utf8]{inputenc}
\usepackage{listings}
\usepackage{multirow}
\usepackage{epstopdf}
\usepackage{amsmath}
\usepackage{amssymb} % fuer Mengen \N, Q, C, R
\graphicspath{{./fig/}}


%% Satzspiegel
\setlength{\hoffset}{-1in} \setlength{\textwidth}{18cm}
\setlength{\oddsidemargin}{1.5cm}
\setlength{\evensidemargin}{1.5cm}
\setlength{\marginparsep}{0.7em}
\setlength{\marginparwidth}{0.5cm}

\setlength{\voffset}{-1.9in}
\setlength{\headheight}{12pt}
\setlength{\topmargin}{2.6cm}
   \addtolength{\topmargin}{-\headheight}
\setlength{\headsep}{3.5cm}
   \addtolength{\headsep}{-\topmargin}
   \addtolength{\headsep}{-\headheight}
\setlength{\textheight}{27cm}

%% How should floats be treated?
\setlength{\floatsep}{12 pt plus 0 pt minus 8 pt}
\setlength{\textfloatsep}{12 pt plus 0pt minus 8 pt}
\setlength{\intextsep}{12 pt plus 0pt minus 8 pt}

\tolerance2000
\emergencystretch20pt

%% Text appearence
% English text
\newcommand{\eg}[1]%
  {\selectlanguage{english}\textit{#1}\selectlanguage{austrian}}

\newcommand{\filename}[1]
  {\begin{small}\texttt{#1}\end{small}}

\newcommand\IFT{\unitlength1mm\begin{picture}(10,2) \put (1,1)
{\circle{1.7}} \put(2,1){\line(1,0){5}} \put(8,1)
{\circle*{1.7}}\end{picture}}
\newcommand\FT{\unitlength1mm\begin{picture}(10,2) \put (1,1)
{\circle*{1.7}} \put(2,1){\line(1,0){5}} \put(8,1)
{\circle{1.7}}\end{picture}}

% A box for multiple choice problems
\newcommand{\choicebox}{\fbox{\rule{0pt}{0.5ex}\rule{0.5ex}{0pt}}}

\newenvironment{wahrfalsch}%
  {\bigskip\par\noindent\makebox[1cm][c]{richtig}\hspace{3mm}\makebox[1cm][c]{falsch}
   \begin{list}%
   {\makebox[1cm][c]{\choicebox}\hspace{3mm}\makebox[1cm][c]{\choicebox}}%
   {\setlength{\labelwidth}{2.31 cm}\setlength{\labelsep}{3mm}
    \setlength{\leftmargin}{2.61 cm}\setlength{\listparindent}{0pt}
    \setlength{\itemindent}{0pt}}%
  }
  {\end{list}}

\newcounter{theaufgabe}\setcounter{theaufgabe}{1}
\newenvironment{aufgabe}[1]%
  {\bigskip\par\noindent\begin{nopagebreak}
   \textsf{\textbf{\arabic{theaufgabe}.\thinspace Aufgabe}}\quad
      \textsf{\textit{#1}}\\*[1ex]%
\stepcounter{theaufgabe}\hspace{2ex}\end{nopagebreak}}
  {\par\pagebreak[2]}

% Innerhalb der Aufgaben erfolgt die weitere Unterteilung mittels einer
% enumerate Umgebung, die allerdings a), b),... zaehlen soll.
\renewcommand{\labelenumi}{\alph{enumi})}
\renewcommand{\labelenumii}{\arabic{enumii})}

% A box to tick for everything which has to done
\newcommand{\abgabe}{\marginpar{$\Box$}}
% Margin paragraphs on the left side
\reversemarginpar

% Language for listings
\lstset{language=Vhdl,
  basicstyle=\small\tt,
  keywordstyle=\tt\bf,
  commentstyle=\sl}

% No indention
\setlength{\parindent}{0.0cm}
% Don't number sections
\setcounter{secnumdepth}{0}


%% Beginning of the text

\begin{document}
\selectlanguage{austrian}
\pagestyle{plain}
% This is the header section
  \thispagestyle{empty}
  \noindent
  %\includegraphics[height=2.5cm]{fig/JKULogoFullEnglShort}
  % blagOPP
  \begin{minipage}[b][2.4cm]{1.0\textwidth}  
  \begin{tabular}{l p{11cm} r} 
    \multicolumn{3}{c}{\centering \begin{large}\begin{bf}
  	\textsf{Digitale Signalverarbeitung, WS 2019/20} \end{bf}\end{large} }  
  	 \\
  	\multirow{2}{*}{\includegraphics[height=1.6cm]{fig/JKU_Logo}} 
  	& \centering Fürst Bernhard, k0442418 \\ Sebastian Ortner k01607533\\ Gruppe 34 \vspace{1.3em}  &
    \multirow{2}{*}{\includegraphics[height=1.9cm]{fig/ISP-Logo-color-02}}  \\	
    & \centering \textit{2. Übung} & \\     
    \multicolumn{3}{c}{\centering \begin{large}
    \textit{Komplexe Zahlen, Fourier-Analyse, LTI Systeme und Abtastung}%
    \end{large} }  
 
  \end{tabular} 
  \end{minipage}
%  \vspace{-1.2em}

  \noindent \rule[0.8em]{\textwidth}{0.12mm}\\[-0.5em]

%%%%%%%%%%%%%%%%%%%%%%%%%%%%%%%%%%%%%%%%%%%%%%%%%%%%%%%%%%%%%%%%%%%%%%%%%%%%%%

\begin{aufgabe}{}
    \begin{enumerate}
      \item \begin{itemize}
           \item{$c_5$} \begin{align*}
            c_1 &= -3 + j5 \\
            c_2 &= \sqrt{2} \exp^{-j\frac{3\pi}{4}} \\
            c_2 &= |c2|\cos{\frac{-3\pi}{4}} + j|c2|\sin{\frac{-3\pi}{4}}\\
            c_2 &= \sqrt{2}*\frac{-1}{\sqrt{2}} + j*\sqrt{2}\frac{-1}{\sqrt{2}} \\
            c_2 &= -1 - j\\
            c_5 &= c_1 + c_2 \\
            c_5 &= (-3 + 5j) + (-1-j) \\
            c_5 &= (-3 + j5) + (-1 -j) \\
            c_5 &= -4 + 4j 
            \end{align*} 
        \item{$c_6$} \begin{align*}
            c_6 &= c_1 -c_2 \\
            c_6 &= (-3+ 5j) - (-1-j) \\
            c_6 &= -2 + 6j \\
            \end{align*}
        \item{$c_7$} \begin{align*}
            c_7 &= c_1 * c_2 \\
            c_1 &= \sqrt{(-3^{2} + 5^{2})} \exp^{\arctan\frac{5}{3}*j} \\
            c_1 &= \sqrt{34}e^{j121^\circ}*\sqrt{2}e^{-j135^\circ} \\
            c_7 &= \sqrt{78}\exp^{-j14^\circ} \implies \sqrt{78}\exp^{j346^\circ} \\
            \end{align*}
        \item{$c_8$}\begin{align*}
            c_8 &= |c_2| \implies \sqrt{2} \\
            \end{align*}
        \item{$c_9$} \begin{align*}
            c_9 &= |c_3|^2\\
            c_9 &= \sqrt{(\frac{1}{\sqrt{2}^{2}})+(\frac{1}{\sqrt{2}^{2}})}^{2} \\
            c_9 &= \frac{1}{2} + \frac{1}{2} \\ 
            c_9 &= 1\\
        \end{align*}
        \item{$c_{10}$}\begin{align*}
            c_{10} &= \arctan(\frac{3}{1} ) = 71.57^\circ \\
        \end{align*}
        \item{$c_{11}$}\begin{align*}
            c_{11} &= \frac{-3 + 5j}{-1 -j}\\
            c_{11} &= \frac{-3(-1) + (5)(-1)}{(-1)^{2} + (-1)^{2}} + j \frac{5(-1)-(-3(-1))}{(-1)^{2} + ((-1)^{2})}\\
            c_{11} &= \frac{3-5}{2} + j\frac{-5-3}{2}\\
            c_{11} &= -1 + (-4j)   
        \end{align*}
       \end{itemize}
       
       \item Siehe dsv2\_1.m 
        ~

       \hspace{-2cm}
       \scalebox{0.4}{
       \includegraphics{Aufgabe2_1_b.png}
       }
       \item Siehe dsv2\_1.m
       
       ~
        \hspace{-3cm}
        \scalebox{0.5}{
        \includegraphics{Aufgabe2_1_c.png}
        }
       \item \begin{align*}
        \alpha_1 &= 45\circ = \frac{45*\pi}{180} = 25\circ = \frac{\pi}{4}= \\
        \alpha_2 &= -90\circ = \frac{(360-90)*\pi}{180} = \frac{270*\pi}{180} = \frac{3\pi}{2} rad \\
        \alpha_3 &= \frac{\pi}{4} = \frac{180\circ}{4} = 45 \circ\\ 
        oder siehe ~\alpha_1\\
        \alpha_4 &= \frac{7\pi}{3} = \frac{\frac{7\pi}{3}180}{\pi} = \frac{7 * 180}{3} = 440 \circ
       \end{align*}
    \end{enumerate}
\end{aufgabe}

\pagebreak
\begin{aufgabe}{}

\begin{enumerate}
  \item{}
~

\hspace{-3.5cm}
\scalebox{0.55}{
\includegraphics{Aufgabe2_2_a.png}
}
\item Die Koeffizienten $a_k$ und $b_k$ skalieren die Kosinus ($a_k$) bzw. die Sinusanteile der k-ten Harmonischen Schwingung einer durch eine Fourierreihe dargestellten Schwingung.

        Die Koeffizienten sind in der gegebenen Gleichung ablesbar. Die Gleichung hat die allgemeine Form:

        \begin{align*}
            x(t) = \frac{a_0}{2} + a_{k_1}*\cos(2\pi k_1 f_0 t) + a_{k2}*\cos(2\pi k_2 f_0 t) +a_{k3}*\cos(2\pi k_3 f_0 t) ...
        \end{align*}

        Wir können also anhand des ganzzahligen Faktors im Argument des Kosinus das $k$ auf das sich der jeweilige Faktor $a_k$ bezieht, feststellen, indem wir mit 2 dividieren da $2\pi$ ja erhalten bleiben muss.

        Die so festgestelleten Faktoren lauten demnach :

        \begin{align*}
            a_0 &= -6 \\
            a_1 &= -3 \\
            b_2 &= -2 \\
            b_{10} &= 2
        \end{align*}

\item
        Die komplexe Form der Fourierreihe ist auch hier endlich und wir können die einzelnen Faktoren der Reihe mit den Formeln :

        \begin{align*}
            c_0 &= \frac{a_0}{2} \\
            c_k &= \frac{a_k -i* b_k}{2}\\
            c_{-k} &= \frac{a_k +i* b_k}{2}
        \end{align*}

        feststellen. Bei fehlendem $a_k$ zu einem vorhandenen $b_k$ muss der fehlende Faktor 0 sein damit der Term in der endlichen Darstellung der Fourierreihe wegfällt et vice versa.

        Die errechneten komplexen Fourierkoeffizienten lauten :

        \begin{align*}
            c_0 &= -3\\
            c_1 &= -1.5\\
            c_{-1} &= -1.5\\
            c_2 &= i\\
            c_{-2} &= -i\\
            c_{10} &= -i\\
            c_{-10} &= i
        \end{align*}

        Visualisiert nach den Vorgaben der Angabe ergibt sich folgendes Bild :

        \hspace{-3cm}
        \scalebox{0.5}{
        \includegraphics{Aufgabe2_2_c.png}
        }
\item
        Die Frequenz hat keinen Einfluss auf $a_k$ und $b_k$ und somit auch keine Einfluss auf die entsprechenden $c_k / c_{-k}$.
\end{enumerate}
\end{aufgabe}
\end{document}
